\documentclass[11pt]{article}
\usepackage[utf8]{inputenc}  % gebruik de juiste 'character encoding'
\usepackage[dutch]{babel}    % definitie van de taal (Engels is de standaard)
\usepackage{hyperref}        % geef URLs netjes weer
\usepackage{booktabs}		 % mooiere tabellen
\usepackage{a4wide}          % papierformaat en marges
\usepackage{graphicx} 		 % Invoegen van plaatjes , ref: https://nl.sharelatex.com/learn/Inserting_Images
\usepackage{listings}
\usepackage{wrapfig}
\graphicspath{ {image/} }   % zet het pad voor de plaatjes
\pagestyle{plain}            % zet alleen paginanummering aan
% titel en auteur van het document
\title{Notulen Project Domotica 1.2}
\date{\today} %dit laat de datum zien
 %BIJ NIEUWE NOTULEN DIT AANZETTEN
 
% einde definities; start van de tekst
\begin{document}
\thispagestyle{empty}
\maketitle %Maak het titelblad
\begin{figure}[h]
	\includegraphics[width=\textwidth]{notulen}

\end{figure}
\newpage
\paragraph{Aanwezigen}

\begin{itemize}
	\item Tycho Meijering
	\item Niels Turenhout
	\item Timo van der Steenhoven
\end{itemize}
Deze vergadering verloopt via Discord [Chatprogramma] vanwege
weersomstandigheden.

\paragraph{Actielijst vorige notulen}
\begin{itemize}
\item Databases? (TIMO)
\item PVA op BB? (NIELS)
\item Communiceren met server? (NIELS)
\item Tegeltjes indeling/schalen? (TYCHO)
\item Communicatie en besturing DaHause (TYCHO)
\item Database bouwen + Connectie (TIMO)
\item Muziekbot/Muziekspeler (NIELS)
\item Creditcounter + unlocken spelletjes (TIMO)
\item Michael bay mode
\item Minesweeper URL
\end{itemize}

\paragraph{Planning}
\begin{itemize}
	\item begin
	\item Planning
	\item Databases?
	\item Niels en \LaTeX => Onderzoeksverslag
	\item Visualstudio project
	\item Deadlines opstellen
	\item WVTTK
	\item afsluiten
	
\end{itemize}

\paragraph{Nieuwe Actielijst}
\begin{itemize}
	\item 
\end{itemize}

















\end{document}