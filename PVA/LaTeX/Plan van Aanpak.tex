\documentclass[11pt]{article}
\usepackage[utf8]{inputenc}  % gebruik de juiste 'character encoding'
\usepackage[dutch]{babel}    % definitie van de taal (Engels is de standaard)
\usepackage{hyperref}        % geef URLs netjes weer
\usepackage{booktabs}		 % mooiere tabellen
\usepackage{a4wide}          % papierformaat en marges
\usepackage{graphicx} 		 % Invoegen van plaatjes , ref: https://nl.sharelatex.com/learn/Inserting_Images
\usepackage{listings}
\usepackage{wrapfig}
\graphicspath{ {images/} }   % zet het pad voor de plaatjes
\pagestyle{plain}            % zet alleen paginanummering aan
% titel en auteur van het document
\title{Plan van Aanpak Project Domotica 1.2}

 \date{} % de datum van het document 
% einde definities; start van de tekst
\begin{document}
\thispagestyle{empty}
\maketitle %Maak het titelblad

\begin{figure}[h]
	\centering
	\includegraphics[width=\textwidth]{inholland}
\end{figure}

\vspace{20mm} %ruimte creëren tussen figuur en opdrachtgever

\center Dit project is in opdracht van Hogeschool INHolland. \center Gemaakt in \LaTeX 

	\vspace{10mm}

\begin{tabular} {l c c} %https://nl.sharelatex.com/learn/Tables#!#Tables_with_fixed_length
	
	Tycho Meijeren & 616047& 616047@student.INholland.nl\\
	
	Timothy van der Steenhoven & 522397 & 522397@student.INholland.nl\\
	
	Niels Turenhout & 614314 & 614314@student.INholland.nl
\end{tabular}


\newpage

\tableofcontents
\thispagestyle{empty}

\newpage

\section[Voorkennis]{Voorkennis van de Projectgroep}
\newpage
\section{De Opdracht}
\newpage
\section{Projectactiviteiten}
\newpage
\section[Grenzen] {Grenzen van het project}
\newpage
\section{Deelproducten}
\newpage
\section{Kwaliteitsbewaking}
\newpage
\section{De Projectorganisatie}
\newpage
\section{Planning}
\newpage
\section{Kosten en baten}
\newpage
\section{Risico's}
\newpage
\section{Bronnen}
\newpage






















\end{document}