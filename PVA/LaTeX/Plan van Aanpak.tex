\documentclass[11pt]{article}
\usepackage[utf8]{inputenc}  % gebruik de juiste 'character encoding'
\usepackage[dutch]{babel}    % definitie van de taal (Engels is de standaard)
\usepackage{hyperref}        % geef URLs netjes weer
\usepackage{booktabs}		 % mooiere tabellen
\usepackage{a4wide}          % papierformaat en marges
\usepackage{graphicx} 		 % Invoegen van plaatjes , ref: https://nl.sharelatex.com/learn/Inserting_Images
\usepackage{listings}
\usepackage{adjustbox}		 % als table verder gaat dan textwidth
\usepackage{wrapfig}
\graphicspath{ {images/} }   % zet het pad voor de plaatjes
\pagestyle{plain}            % zet alleen paginanummering aan
% titel en auteur van het document
\title{Plan van Aanpak Project Domotica 1.2}

 \date{} % de datum van het document 
% einde definities; start van de tekst
\begin{document}
\thispagestyle{empty}
\maketitle %Maak het titelblad

\begin{figure}[h]
	\centering
	\includegraphics[width=\textwidth]{inholland}
\end{figure}

\vspace{20mm} %ruimte creëren tussen figuur en opdrachtgever

\center Dit project is in opdracht van Hogeschool INHolland. \center Gemaakt in \LaTeX 

	\vspace{10mm}

\begin{tabular} {l c c} %https://nl.sharelatex.com/learn/Tables#!#Tables_with_fixed_length
	
	Tycho Meijeren & 616047& 616047@student.INholland.nl\\
	
	Timothy van der Steenhoven & 522397 & 522397@student.INholland.nl\\
	
	Niels Turenhout & 614314 & 614314@student.INholland.nl
\end{tabular}


\newpage

\tableofcontents
\thispagestyle{empty}

\newpage

\section[Voorkennis]{Voorkennis van de Projectgroep}
\newpage
\section{De Opdracht}
\newpage
\section{Projectactiviteiten}

\begin{flushleft}
	
	In dit project komen de volgende activiteiten naar voren:
	
	
	\begin{itemize}
		
		\item Uitzoeken github
		\item Maken Samenwerkingsovereenkomst
		\item Maken PvA
		\item Maken ERD, PSD's
		\item Realisiatie webdaschboard
		\item Maken onderzoeksrapport
	\end{itemize}
	
\end{flushleft}

	
	

\newpage
\section[Grenzen] {Grenzen van het project}

In dit project hebben de studenten niet al te veel grenzen om rekening mee te houden. Het mag zo uitgebreid al ze willen zolang ze maar de eisen er in hebben verwerkt. Ook kunnen de studenten er voor kiezen alleen de eisen te verwerken hiermee zal de project groep wel een voldoende halen, maar als ze voor een hoger cijfer willen gaan moeten ze toch wat extra's doen. 

\newpage
\section{Deelproducten}
\newpage
\section{Kwaliteitsbewaking}
\newpage
\section{De Projectorganisatie}
\newpage
\section{Planning}

\begin{adjustbox}{width=\textwidth}
\begin{tabular}{|l | c | r |}
	\hline
	Lesweek & Deliverable & Activiteiten \\ \hline
	1 & -	& Projectgroep formeren en opstellen samenwerkingsovereenkomst \\ \hline
	2 & Samenwerkingsovereenkomst & Plan van Aanpak\\ \hline
	3 & - & PvA bijschaven en begin maken aan ontwerpmodellen
	(PSD’s, ERD)\\ \hline
	4 & PvA & Ontwerpmodellen opstellen\\ \hline
	5 & ERD en PSD's & Realisatie van het webdashboard\\ \hline
	6 & - & Realisatie voortzetten\\ \hline
	7 & - & Lesluwe week (feestdagen): Begin maken aan rapportage.\\ \hline
	8 & - & Afronden rapportage en realisatie.\\ \hline
	9 & - & Voorbereiden op projectassessment en demonstratie.\\ \hline
	10 & Definitieve versie van	het rapport, demonstratie.&
Projectassessment.\\ \hline
	
\end{tabular}
\end{adjustbox}
\newpage
\section{Kosten en baten}
\newpage
\section{Risico's}
\newpage
\section{Bronnen}
\newpage






















\end{document}