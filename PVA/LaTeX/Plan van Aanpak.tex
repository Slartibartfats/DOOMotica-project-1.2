<<<<<<< HEAD
\documentclass[11pt]{article}
\usepackage[utf8]{inputenc}  % gebruik de juiste 'character encoding'
\usepackage[dutch]{babel}    % definitie van de taal (Engels is de standaard)
\usepackage{hyperref}        % geef URLs netjes weer
\usepackage{booktabs}		 % mooiere tabellen
\usepackage{a4wide}          % papierformaat en marges
\usepackage{graphicx} 		 % Invoegen van plaatjes , ref: https://nl.sharelatex.com/learn/Inserting_Images
\usepackage{listings}
\usepackage{adjustbox}		 % als table verder gaat dan textwidth
\usepackage{wrapfig}
\graphicspath{ {images/} }   % zet het pad voor de plaatjes
\pagestyle{plain}            % zet alleen paginanummering aan
% titel en auteur van het document
\title{Plan van Aanpak Project Domotica 1.2}

 \date{} % de datum van het document 
% einde definities; start van de tekst
\begin{document}
\thispagestyle{empty}
\maketitle %Maak het titelblad

\begin{figure}[h]
	\centering
	\includegraphics[width=\textwidth]{inholland}
\end{figure}

\vspace{20mm} %ruimte creëren tussen figuur en opdrachtgever

\center Dit project is in opdracht van Hogeschool INHolland. \center Gemaakt in \LaTeX 

	\vspace{10mm}

\begin{tabular} {l c c} %https://nl.sharelatex.com/learn/Tables#!#Tables_with_fixed_length
	
	Tycho Meijeren & 616047& 616047@student.INholland.nl\\
	
	Timothy van der Steenhoven & 522397 & 522397@student.INholland.nl\\
	
	Niels Turenhout & 614314 & 614314@student.INholland.nl
\end{tabular}


\newpage

\tableofcontents
\thispagestyle{empty}

\newpage

\section[Voorkennis]{Voorkennis van de Projectgroep}
\newpage
\section{De Opdracht}
\newpage
\section{Projectactiviteiten}

\begin{flushleft}
	
	In dit project komen de volgende activiteiten naar voren:
	
	
	\begin{itemize}
		
		\item Uitzoeken github
		\item Maken Samenwerkingsovereenkomst
		\item Maken PvA
		\item Maken ERD, PSD's
		\item Realisiatie webdaschboard
		\item Maken onderzoeksrapport
	\end{itemize}
	
\end{flushleft}

	
	

\newpage
\section[Grenzen] {Grenzen van het project}
%grenzen 1
%In dit project hebben de studenten niet al te veel grenzen om rekening mee te houden. Het mag zo uitgebreid al ze willen zolang ze maar de eisen er in hebben verwerkt. Ook kunnen de studenten er voor kiezen alleen de eisen te verwerken hiermee zal de project groep wel een voldoende halen, maar als ze voor een hoger cijfer willen gaan moeten ze toch wat extra's doen.

%grenzen 2
%Dit project grenst aan meerdere vakken van de leerlingen waar onder databases ,web programmeren en onderzoek. verder moet het project ook grenzen aan de domotica kant van de opleiding dit door middel van het hulpmiddel dahaus. Het verslag dat de leerlingen zullen schrijven word ook deels gecontroleerd door hun onderzoek leraar. De onderdelen databases en web programmeren worden binnen het project beoordeeld door de project begeleiders. 

\newpage
\section{Deelproducten}
\newpage
\section{Kwaliteitsbewaking}
\newpage
\section{De Projectorganisatie}
\newpage
\section{Planning}

\begin{adjustbox}{width=\textwidth}
\begin{tabular}{|l | c | r |}
	\hline
	Lesweek & Deliverable & Activiteiten \\ \hline
	1 & -	& Projectgroep formeren en opstellen samenwerkingsovereenkomst \\ \hline
	2 & Samenwerkingsovereenkomst & Plan van Aanpak\\ \hline
	3 & - & PvA bijschaven en begin maken aan ontwerpmodellen
	(PSD’s, ERD)\\ \hline
	4 & PvA & Ontwerpmodellen opstellen\\ \hline
	5 & ERD en PSD's & Realisatie van het webdashboard\\ \hline
	6 & - & Realisatie voortzetten\\ \hline
	7 & - & Lesluwe week (feestdagen): Begin maken aan rapportage.\\ \hline
	8 & - & Afronden rapportage en realisatie.\\ \hline
	9 & - & Voorbereiden op projectassessment en demonstratie.\\ \hline
	10 & Definitieve versie van	het rapport, demonstratie.&
Projectassessment.\\ \hline
	
\end{tabular}
\end{adjustbox}
\newpage
\section{Kosten en baten}
\newpage
\section{Risico's}
\newpage
\section{Bronnen}
\newpage






















=======
\documentclass[11pt]{article}
\usepackage[utf8]{inputenc}  % gebruik de juiste 'character encoding'
\usepackage[dutch]{babel}    % definitie van de taal (Engels is de standaard)
\usepackage{hyperref}        % geef URLs netjes weer
\usepackage{booktabs}		 % mooiere tabellen
\usepackage{a4wide}          % papierformaat en marges
\usepackage{graphicx} 		 % Invoegen van plaatjes , ref: https://nl.sharelatex.com/learn/Inserting_Images
\usepackage{listings}
\usepackage{wrapfig}
\graphicspath{ {images/} }   % zet het pad voor de plaatjes
\pagestyle{plain}            % zet alleen paginanummering aan
% titel en auteur van het document
\title{Plan van Aanpak Project Domotica 1.2}

 \date{} % de datum van het document 
% einde definities; start van de tekst
\begin{document}
\thispagestyle{empty}
\maketitle %Maak het titelblad

\begin{figure}[h]
	\centering
	\includegraphics[width=\textwidth]{inholland}
\end{figure}

\vspace{20mm} %ruimte creëren tussen figuur en opdrachtgever

\centering Dit project is in opdracht van Hogeschool INHolland.\\ Gemaakt in \LaTeX 

	\vspace{10mm}

\begin{tabular} {l c c} %https://nl.sharelatex.com/learn/Tables#!#Tables_with_fixed_length
	
	Tycho Meijeren & 616047& 616047@student.INholland.nl\\
	
	Timothy van der Steenhoven & 522397 & 522397@student.INholland.nl\\
	
	Niels Turenhout & 614314 & 614314@student.INholland.nl
\end{tabular}


\newpage

\tableofcontents
\thispagestyle{empty}

\newpage
\section{Achtergrond}

\paragraph{}
\begin{flushleft}
In dit hoofdstuk komt, zoals de titel al aangeeft, de achtergrond van het project aan bod.
De achtergrond zal bestaan uit een korte uitleg over de opdrachtgever van het project, de 
setting waarin het project uitgevoerd wordt, een korte uitleg over de vormgeving van het project
en een beschrijving van de opkomende hoofdstukken.
\end{flushleft}
\paragraph{'Doomotica'}
\begin{flushleft}
	 Dit project heeft de naam Doomotica gekregen, vanwege een typfout en, omdat de projectgroep dat een passende naam vond. 
	De opdrachtgever van het project is Hogeschool INHolland te Alkmaar vanuit de opleiding Technische Informatica, 
	jaar 1 periode 2. De contactpersonen vanuit school zijn:\\
	\begin{tabular}{l l l}
		Docent TI & Sander Gieling & Sander.Gieling@inholland.nl\\
		Docent TI & Martijn Oldenburg & Martijn.Oldenburg@inholland.nl\\
		\end{tabular}
	\end{flushleft}

\begin{flushleft}
in januari 2018 zal door middel van een assessment het project beoordeeld worden door bovenstaande opdrachtgevers.
\end{flushleft}
\paragraph{Setting}
\begin{flushleft}
	De setting van het project valt binnen de eisen die gesteld zijn in de projecthandleiding (Jeroen Bol \& Sander Gieling, 2015). De aanleiding van dit project is hierin ook te vinden: \\
	"Tijdens  deze  projectperiode  is  het  de  bedoeling  dat  studenten  in  het  kader  van
	Web	Development een werkende webapplicatie (gericht op het mobiele platform) ontwerpen
	en  realiseren." (Jeroen Bol \& Sander Gieseling, 2015)
\end{flushleft}

\begin{flushleft}
	Het bovenstaande citaat is breed. Om de vormgeving van het project helderder te krijgen volgen hier een aantal concrete eisen vanuit de handleiding: 
	\begin{itemize}
		\item Een web-applicatie die in een door de projectgroep gekozen browser gebruikt kan worden.
		\item Een communnicatie met door de projectgroep gekozen database voor het opslaan en uitlezen van gegevens server-side.
		\item Voor elk database systeem dient er een Entity-relationshipmodel (ERD) gemaakt te worden. 
		\item Voor elke functionaliteit dient er een Programma Structuur Diagram gemaakt te worden.
		\item De web-applicatie moet de geleverde domotica simulator 'DaHause' kunnen \underline{besturen}.  
		\end{itemize}
	In een later hoofdstuk in het plan van aanpak zal de projectgroep het project nauwkeuriger afbakenen.

	\end{flushleft}

\begin{flushleft}
	Dit sluit het huidige hoofdstuk af. Het volgende hoofdstuk gaat verder in op de eisen die school stelt en hoe de projectgroep daar vorm aan gaat geven.
\end{flushleft}



\newpage
\section{De Opdracht}
\paragraph{}
\begin{flushleft}
	In dit hoofdstuk wordt het project verder afgebakend en worden er een aantal keuzes van de projectgroep toegelicht met betrekking tot het gebruik van programma's. De definitieve keuzes voor vormgeving en uiteindelijke resultaten zullen in het onderzoeksrapport staan.
\end{flushleft}

	\paragraph{Verslaglegging en Opslag}
	\begin{flushleft}
	De verslaglegging van het project zal volledig in \LaTeX \space gebeuren. Applicatie-bestanden worden op GitHub (\textbf{TO DO BRONVERMELDING}) opgeslagen, zodat de projectgroep onderling project-bestanden kunnen uitwisselen en tegelijkertijd voorzien van commentaar.
	\end{flushleft}
\paragraph{Modellen}
\begin{flushleft}

\end{flushleft}

	\paragraph{Web-applicatie}
	\begin{flushleft}
	De web-applicatie zal gebouwd worden in ASP.NET 4.0. Deze versie wordt aangeraden vanuit school, omdat in deze versie het integreren van Javascript relatief weinig stappen betreft (kopiëren en plakken in veel gevallen). De web-applicatie zal gebouwd worden met behulp van Visual Studio 2017. %TO DO BRONVERMELDING%
	\end{flushleft} 
\paragraph{Database}
\begin{flushleft}
	
\end{flushleft}

\newpage
\section{Projectactiviteiten}
\newpage
\section[Grenzen] {Grenzen van het project}
\newpage
\section{Deelproducten}
\newpage
\section{Kwaliteitsbewaking}
\newpage
\section{De Projectorganisatie}
\newpage
\section{Planning}
\newpage
\section{Kosten en baten}
\newpage
\section{Risico's}
\newpage
\section{Bronnen}
\begin{itemize}
	\item Jeroen Bol \& Sander Gieling. (5 nov 2015). \textit{Projecthandleiding Domotica Technische Informatica, periode 2 jaar 1} (Inholland University of Applied Sciences, Alkmaar).  
\end{itemize}























>>>>>>> Timo's-kutdingen
\end{document}