\documentclass[11pt]{article}
\usepackage[utf8]{inputenc}  % gebruik de juiste 'character encoding'
\usepackage[dutch]{babel}    % definitie van de taal (Engels is de standaard)
\usepackage{hyperref}        % geef URLs netjes weer
\usepackage{booktabs}		 % mooiere tabellen
\usepackage{a4wide}          % papierformaat en marges
\usepackage{graphicx} 		 % Invoegen van plaatjes , ref: https://nl.sharelatex.com/learn/Inserting_Images
\usepackage{listings}
\usepackage{wrapfig}
\graphicspath{ {images/} }   % zet het pad voor de plaatjes
\pagestyle{plain}            % zet alleen paginanummering aan
% titel en auteur van het document
\title{Onderzoeks verslag DOOMotica}

\begin{document}
	\maketitle % maakt de title
	\begin{figure}[h]
		\centering
		\includegraphics[width=\textwidth]{inholland}
	\end{figure}
	
	\newpage
	
	\tableofcontents
	\newpage
	\section{inleiding}
	De reden dat we dit project doen is zodat we de domotica in een huis zouden kunnen besturen. Hier voor maken we een webpagina waarop op onder andere de functies van Dahaus, spelletjes, tegeltjes voor je favoriete sites en nog wat andere tegeltjes voor je site geschiedenis.
	\newpage
	
	\section{Opdracht}
	De opdracht voor de studenten is het maken van een website waar de gebruiker een account kan aanmaken en op dit account kan inloggen. Na het inloggen moet de gebruiker zijn favorieten site op de pagina kunnen zetten in de daar aangewezen tegels er voor. verder zijn er nog tegels voor de laatst bezochte sites. verder is er nog plek voor simpele spelletjes en als laatst moet de gebruiker Dahaus kunnen gebruiken.
	\newpage
	
	\section{Probleemstelling}
	
	\newpage
	
	\section{Centrale vraag}
	
	\newpage
	
	\section{Deelvragen en onderzoeksvragen}
	\subsection{Dahaus}
	\subsection{spelltjes}
	De spelletjes die de studenten op de site moeten verwerken zijn simpele spelletjes voor de gebruiker om te spelen op de webpagina zelf. Er kunnen spelletjes op staan zoals: 
	\subsection{Tegeltjes}
	De Tegeltjes die op de site verwerkt zitten zijn voor de gebruiker. Om zelf in te vullen met zijn of haar eigen favorieten websites.
	\newpage
	
	\section{Verantwoording gebruikte methodes}
	
	\newpage
	
	\section{Analyse van de gegevens}
	\subsection{Dataverzameling}
	\subsection{Analyse}
	\subsection{Deelconclusie/ aanbeveling}
	
	\newpage
	
	\section{Eindconclusie}
	
	\newpage
	
	\section{Aanbeveling}
	
	\newpage
	
	\section{Literatuurlijst}
	
	\newpage
\end{document}